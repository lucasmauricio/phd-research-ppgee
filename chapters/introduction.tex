\chapter{Introduction} \label{chap:introduction}


\section{Justificativa}

- a melhoria do transito pode ajudar a salvar vidas

- a melhoria do transito pode ajudar a economia (um carro particular fica 95\% do seu tempo de vida sem uso) -- ver trabalho do prof. Alexander Wyglinski (ele falou isso no WCNPS).




\section{Objectives}

\todo[inline]{organizar essa seção da mesma forma que fiz no mestrado}

the main objective of this thesis is to
\textbf{
propor um modelo de interação entre a estrada e outros veículos para dar suporte a direção
}
-- não é para ir na direção do self-driving porque não haverá como trabalhar isso aqui no laboratório.


To achieve this main goal, several secondary objectives have been defined:

\todo[inline]{verificar se tem um objetivo intermediário para cada problema levantado na motivação.}

\begin{itemize}
    \item ESCREVER

\end{itemize}



\section{Research Method}  \label{sec:intro-method}

\todo[inline]{dizer que já existia o Raise como ponto de partida?}



\section{Outline} \label{sec:intro-outline}

Como pretendo organizar a pesquisa:

\begin{itemize}
    \item[parte 1:] falar da aplicação do middleware da minha dissertação no cenário automobilístico;
    \item[parte 2:] aplicar tecnologia de comunicação com o ambiente ao middleware do carro;
    \item[parte 3:] aplicar tecnologia de V2V;
    \item[parte 4:] discutir o algoritmo de inteligência;
    \item[parte 5:] apresentar o aplicativo mobile que interage com a central multimídia do carro;
\end{itemize}

